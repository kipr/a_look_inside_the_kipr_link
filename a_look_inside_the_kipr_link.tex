\documentclass[12pt,letterpaper]{article}

\usepackage[pdftex]{graphicx}
\usepackage{float}
\usepackage{hyperref}
\usepackage{tipa}
\usepackage{natbib}
\usepackage[american]{babel}
\usepackage[margin=1in]{geometry}
\usepackage{fancybox}
\usepackage[usenames,dvipsnames,svgnames,table]{xcolor}
\usepackage{wrapfig}
\usepackage{textcomp}
\usepackage{amssymb}

\pagestyle{empty}

\newcommand{\urlfootnote}[1]{\footnote{\url{#1}}}
\newcommand{\code}[1]{\par\texttt{#1}\par}
\newcommand{\authorInfo}[1]{\gdef\authInfo{#1}}

\title{A Look Inside the KIPR Link}
\author{Braden McDorman, Joshua Southerland}
\authorInfo{KISS Institute for Practical Robotics, University of Oklahoma \\
	\texttt{bmcdorman@kipr.org}, \texttt{southou@gmail.com}}
 
\fboxrule=2pt
\newcommand{\bcolorbox}[4]{\noindent \\ \fcolorbox{#1}{#2} {\parbox{\textwidth}{\vspace{.1em}\textbf{#3} #4\vspace{.1em}}} \\}
\newcommand{\whoathere}[1]{\bcolorbox{red}{MistyRose}{Hold On:}{#1}}
\newcommand{\note}[1]{\bcolorbox{yellow}{Seashell}{Note:}{#1}}
\newcommand{\why}[1]{\bcolorbox{black}{GhostWhite}{Why?}{#1}}
\newcommand{\novice}[1]{\bcolorbox{green}{Honeydew}{Novice's Corner:}{#1}}
\newcommand{\getit}[1]{\bcolorbox{Indigo}{Lavender}{Get It!}{#1}}

\makeatletter


\begin{document}
	
	\noindent\begin{minipage}[l]{.8\textwidth}\begin{flushleft}\begin{small}
	\textbf{\@title}

	\@author

	\authInfo
	\end{small}\end{flushleft}\end{minipage}\vspace{.25in}
	
	\begin{center}
	\begin{Large}
	\textbf{\@title}
	\end{Large}
	\end{center}
	
	\tableofcontents
	
	\section{Introduction}
	The KIPR Link (herein referred to as just ``The Link'') is the fifth generation educational robotics controller used in the Botball robotics
	competition. The Link is also the first robotics controller designed from the ground up for Botball. Since the Link runs a full linux kernel
	and is completely open source, it is a perfect platform for modification and experimentation by more advanced users. This document hopes
	to serve as a guide for the curious and ambitious user that wishes to modify the Link for their own purposes.
	
	\subsection{Disclaimer}
	Modifying the Link's firmware can lead to system instabilities and possible bricking. While this article does provide documentation for
	un-bricking any Link manually, these steps require opening the case and \textbf{will void your warranty}.
	
	\subsection{Document Format}
	
	Colored boxes are placed throughout this document to provide caution and insight to the reader:
	
	\whoathere{Read everything in red boxes \textbf{very carefully}.}
	\note{Take the following information under advisement.}
	\why{Provides insight into why certain design decisions were made.}
	\getit{Provides links to documentation and source files.}
	\novice{A high-level overview of a possibly foreign concept, component, or system. These boxes might also
	contain links to selected readings.}
	
	At the end of every section, I will attempt to provide relevant links to documentation, source code, and other useful references
	for beginning development with the Link.
	
	\section{System Overview}
	
	\subsection{Hardware Overview}
	
	While this document is primarily aimed at understanding and modifying the Link's software, a basic understanding of
	the Link's hardware is essential.
	
	\begin{figure}[H]
		\begin{center}
			\includegraphics[width=\textwidth,height=\textheight,keepaspectratio]{hardware_overview.pdf}
			\caption{Hardware Overview of the KIPR Link}
		\end{center}
	\end{figure}
	
	The Link features an ARMv5te CPU and a Spartan 6 FPGA. Both of these chips work in tandem to enable all of the functionality
	of the Link.
	
	\novice{A FPGA, or Field-Programmable Gate Array, is an integrated circuit that can be programmed to perform highly parallel and fast
	custom logic. FPGAs can be orders of magnitude faster than CPUs for certain types of operations, but are not designed to handle the
	sequential logic that a CPU would normally perform.}
	
	Unfortunately the FPGA is more of a hindrance than a benefit in the current version of the Link. The FPGA was originally added to
	mediate fast vision processing like the XBC's. Using a USB camera, however, means that the cost of transporting the image data to the
	FPGA is more expensive than just doing the computation on the CPU. To realize the performance the XBC enjoyed, the camera would need
	a direct (and preferably non-USB) connection to the FPGA.
	
	\subsection{Software Overview}
	
	Most of the software in this document was designed and implemented when the Link was still called by its codename: ``Kovan''.
	
	\why{Kovan is the name of our hardware designer's bus stop in Singapore. That's all I've got.}
	
	\begin{figure}[H]
		\begin{center}
			\includegraphics[width=\textwidth,height=\textheight,keepaspectratio]{software_overview.pdf}
			\caption{Software Overview of the KIPR Link}
		\end{center}
	\end{figure}
	
	\section{The Software Stack}
	
	The Link's full software stack is composed of hundreds of libraries, executables, and configuration files. Fortunately, many of these
	are not entirely relevant to the discussion of the Link's software components. In this section I will enumerate the important libraries
	and executables to understand, provide notes on their implementation, provide examples of their usage, and describe how a developer
	might go about modifying these packages.
	
	\subsection{\texttt{libkovanserial}}
	\texttt{libkovanserial} is the unified network and USB communication protocol for the Link and KISS IDE.
	\texttt{libkovanserial} is divided into three layers:
	\begin{enumerate}
		\setlength{\itemsep}{0em}
		\item ``Transmitters'' are back-ends that implement a specific communication mechanism, such as TCP/IP sockets or USB comm ports.
		Security is not handled on this layer.
		\item The ``Transport Layer'' handles packet creation, checksumming, and a basic ACK/resend mechanism for non-reliable protocols such
		as serial communication ports. Session-level security is defined on this layer.
		\item The ``Protocol Layer'' helps facilitate protocol-level communication with the KIPR Link. This is intentionally left
		as a somewhat leaky abstraction. User password security is implemented on this layer.
	\end{enumerate}
	
	\subsubsection{Authentication Handshake}
	\texttt{libkovanserial} uses XOR encryption with a mutual shared session key that is negotiated during handshake. XOR encryption is notoriously insecure with plain text, 
	but the use of random data paired with cryptographically secure data makes normal XOR encryption attack vectors useless. Since the session key is a pseudo-randomly generated
	512 bits and sessions are short lived, cracking the session key with brute force is unlikely. \texttt{libkovanserial} also goes a step further and fills empty space in
	packets with pseudo-random bytes. This prevents sniffers from detecting the key using zeroed-out sections of packets. This handshake method guards against
	man-in-the-middle attacks and other sniffers by XOR encrypting the session key during transmission using a private but mutual piece of information:
	\texttt{sha1(password)}. Since the session key is 512 pseudo-random bits and the SHA1 key is a cryptographically strong hash, decoding either piece
	of information is unlikely. A new session key is generated with every high-level command to the Link, so a
	session key does not remain valid for more than a few seconds.
	
	\subsubsection{Handshake Example}
	
	A typical handshake looks like this:
	\begin{enumerate}
		\setlength{\itemsep}{0em}
		\item Ask the server if it requires authentication. If no, \texttt{finish}. If yes, \texttt{goto 2}.
		\item Send the server our password's MD5 hash.
		\item Check if authentication was successful. If no, prompt user for new password and \texttt{goto 2}.
		If yes, decrypt the session key using our password's SHA1 hash and \texttt{finish}.
	\end{enumerate}
	
	
	\subsection{The \texttt{kovan-serial} Daemon}
	The \texttt{kovan-serial} daemon mediates all incoming connections to the Link over USB and Wi-Fi.
	
	\subsubsection{Notes on Using the USB Micro Port}
	The kernel driver used for the USB Micro port on the Link (\texttt{otg\_serial}) is unfortunately very buggy with the Link's hardware.
	If the USB cable is ever physically disconnected and then subsequently reconnected to the Link, the kernel driver enters an error state
	that \emph{can not be directly detected}. Attempting to read or write data while in this error state will eventually lead to the kernel
	driver locking up, which can not be fixed without power cycling the device. After much experimentation, it was discovered that this error
	state can be indirectly detected by attempting to \texttt{write} an array of size zero to the open USB file descriptor at semi-frequent
	intervals (\texttt{kovan-serial} checks every two seconds). If \texttt{write} fails and sets \texttt{errno} to \texttt{EIO}, close the
	file descriptor and re-open it. It should be noted that \texttt{read} \emph{will not return any error} other than setting \texttt{errno} to 
	\texttt{EAGAIN} (An error that means there was no data ready to read, but that the connection is still good), even though attempting to read once
	the error state is entered could eventually result in the kernel driver locking up.
	
	
	\subsection{\texttt{libkovan}}
	\texttt{libkovan} is the most important library on the Link. It is designed to implement and/or expose every piece of functionality a
	user would expect from a robotics controller. This includes, but is not limited to:
	\begin{itemize}
		\setlength{\itemsep}{0em}
		\item Motor actuation
		\item Servo actuation
		\item Camera perception
		\item AR.Drone communication
		\item iRobot Create communication
		\item High-level threading routines
		\item Simple key/value configuration files
		\item Data collection and export
		\item Utility functions
	\end{itemize}
	
	\getit{\url{http://github.com/kipr/libkovan}}
	
	\subsection{\texttt{libkar}}
	\texttt{libkar} (LIBKissARchive) is an extremely simple Qt-based archive format used to store and transport data in KISS IDE and its targets
	(including, but not limited to, the Link.) \texttt{libkar} is implemented as a flat key/value dictionary with methods to treat the
	keys as a two-dimensional tree structure when useful. This dictionary is then serialized to 
	
	\novice{\emph{Serialization} is a fancy term for taking a data structure like a color, time, or image and converting it into an array of bytes
	that can be either stored on a disk or transferred to another computer via a serial connection. Serialization is reversed using \emph{deserialization}.}
	
	(needs cleaning up)
	For more information on how the data is serialized, please see: http://qt-project.org/doc/qt-4.8/datastreamformat.html
	
	\getit{\url{http://github.com/kipr/libkar}}
	\subsection{\texttt{pcompiler}}
	\texttt{pcompiler} (short for precedence compiler) is a library that automatically attempts to produce executables from a set of input files.
	\texttt{pcompiler} uses file suffix precedence to create a weak build ordering, and then applies \emph{transforms} to mutate the input into an
	output. For example, given the files: \texttt{main.c}, \texttt{functions.h}, and \texttt{functions.c}, \texttt{pcompiler} will generate the 
	following transforms (herein denoted by a $\rightarrow$): 
	
	\begin{enumerate}
		\item \{\texttt{functions.h}\} $\rightarrow \emptyset $ (\texttt{passthrough} transform)
		\item \{\texttt{functions.c}, \texttt{main.c}\} $\rightarrow $ \{\texttt{functions.c.o}, \texttt{main.c.o}\}  (\texttt{c} transform)
		\item \{\texttt{functions.c.o}, \texttt{main.c.o}\} $ \rightarrow $ \{\texttt{a.out}\} (\texttt{o} transform)
	\end{enumerate}
	
	\novice{The C language goes through three build phases: preprocess, compile, and link. The preprocess step takes macros like \texttt{\#include "file.h"} and replaces them with the specified file before passing the \texttt{.c} files to the compiler. The compiler generates \texttt{.o} (\emph{object files)} from the \texttt{.c} files. Object files are linked together to create an executable (the default executable name on linux is \texttt{a.out})}
	
	The \texttt{passthrough} transform is used to retain files in the build directory but remove them from the compilation set. Since \texttt{.h} files
	are handled by the \texttt{.c} compiler, there is no need for pcompiler to consider them further. \texttt{pcompiler} expects to only produce one
	output file (or \emph{terminal}), which means keeping the \texttt{.h} around would result in a failed compilation.
	
	\why{While most advanced programmers would prefer to use a \texttt{Makefile} or other build system for their code, the syntax of these files
	are often esoteric and distract from mastering the basics of programming. Since the KIPR Link's target audience is middle and high school
	students, the decision was made to avoid manual build systems altogether}.
	\getit{\url{http://github.com/kipr/pcompiler}}
	\subsection{\texttt{botui}}
	
	\texttt{botui} is the graphical user interface the user is presented upon starting a Link. \texttt{botui} focuses on four primary tasks: Sensor visualization, simple motor/servo control, device configuration, and, perhaps most importantly, managing user programs.
	\getit{\url{http://github.com/kipr/botui}}
	\subsection{\texttt{kovan-kmod}}
	\texttt{kovan-kmod} is a kernel module that mediates communication between \texttt{libkovan} instances and the FPGA. The chosen IPC
	mechanism was UDP, as UDP allows full duplex communication and avoid the synchronization headaches of other IPC mechanisms.
	
	\getit{\url{http://github.com/kipr/kovan-kmod}}
	\subsection{\texttt{kovan-recovery}}
	Recovery mode is a special boot mode that is used to perform firmware upgrades and recover semi-bricked devices.
	To understand  \texttt{kovan-recovery}, it is first necessary to understand the boot process of the Link and the initialization process of linux.
	The Link uses a custom Master Boot Record (MBR) initially designed by Chumby Industries for the Chumby. (cite) This custom MBR
	contains two linux kernels: One regular kernel and one special recovery kernel. It is important to reiterate that these kernels \emph{are not} 
	stored on the device's filesystem, but instead directly in the MBR. The process of booting a linux kernel goes something like this:
	\begin{enumerate}
		\item Load the kernel into RAM.
		\item The kernel then builds what is called an \texttt{initramfs} and executes a process called \texttt{init}. \texttt{initramfs} contains
		an extremely minimal filesystem that is used by the \texttt{init} process. The idea behind an \texttt{initramfs} is that certain operations
		that don't belong in the kernel can be executed before the main filesystem is loaded. For example, imagine the case in which a filesystem
		is encrypted. Linux doesn't know how to use an encrypted filesystem, so a developer could design an \texttt{initramfs} that loads the
		correct keys, decrypts the filesystem, and then continues the boot process. It should be noted that the \texttt{initramfs} is actually
		compiled \emph{directly into} the kernel itself.
		\item The real root filesystem is loaded and initialization of the system continues.
	\end{enumerate}
	
	\texttt{kovan-recovery} takes advantage of \texttt{initramfs} to ``hijack'' the boot process before a root filesystem is mounted. This detail is
	important, as \texttt{kovan-recovery} works by overwriting the entire internal SD card. Since the SD card is being overwritten, it can not be used
	as a root filesystem. Since \texttt{kovan-recovery} and linux exist only in RAM, this is not an issue.
	
	As mentioned earlier, the \texttt{initramfs} is compiled directly into the kernel. This means that the Link needs two kernels: One special
	recovery kernel and one regular kernel. Before the linux kernel is loaded, the side button is checked to see if it is pressed down. If it is,
	the special kernel is loaded instead of the regular one.
	
	\texttt{kovan-recovery} itself is an extremely simple C program that uses \texttt{zlib} and basic file i/o to inflate and subsequently write
	to the internal drive. Since including libraries in the \texttt{initramfs} is burdensome, \texttt{kovan-recovery} performs double-buffered
	drawing directly to the \texttt{/dev/fb} LCD frame-buffer, using an extremely simple bitmapped font to perform font rendering.
	
	\novice{Unix represents almost every installed device as a special type of file in the \texttt{/dev} directory. This means that
	a developer can open a device and write to it like it was a file. As noted above, \texttt{kovan-recovery} opens the LCD as a file and
	writes pixels at certain offsets that correspond to onscreen (x, y) locations. This concept is applied in several places throughout the system, and is important to understand.}
	\getit{\url{http://github.com/kipr/kovan-recovery}}
	
	\subsection{KISS IDE}
	An elementary understanding of KISS IDE 4's architecture is important for modifying and extending the Link.
	
	\begin{figure}[H]
		\begin{center}
			\includegraphics[width=\textwidth]{kiss_ide.pdf}
			\caption{A Hypothetical Runtime Snapshot of KISS IDE}
		\end{center}
	\end{figure}
	
	Two of the core concepts in KISS IDE are that of the \texttt{Interface} and \texttt{Target}. \texttt{Interface}s encapsulate
	the the discovery, enumeration, and creation of \texttt{Target}s for a specific device. A \texttt{Target} is used to \emph{download},
	\emph{compile}, \emph{run}, and perform other operations on a connected device. Another core concept in KISS IDE is that of the
	\texttt{Unit}. A \texttt{Unit} is an abstract class that uses the visitor pattern to create in-memory project and source file archives.
	For example, imagine the existence of a project called ``Task1'' that contains three source files: \texttt{foo.c}, \texttt{bar.h}, and 
	\texttt{baz.c}. 
	Now imagine that the user invokes a \emph{download} command on the source file \texttt{foo.c}. \texttt{foo.c} will ask that its
	\texttt{Unit} invoke a \emph{download} command on the top-level \texttt{Unit}, which is Task1. Task1's unit will then create an empty archive,
	add project files to it, and \emph{visit} its children \texttt{Unit}s recursively. A child \texttt{Unit} may then add its required files
	to the archive. Once this recursive process finishes, the completed archive is sent wrapped in a \texttt{CommunicationEntry} and queued in
	the \texttt{CommunicationManager}. The \texttt{CommunicationManager} then handles non-blocking execution of the task on the given target.
	
	
	The most important thing to note here is that KISS IDE \emph{does not} compile source files locally when communicating with a Link.
	A \emph{compile} command results in the following chain of actions:
	\begin{enumerate}
		\setlength{\itemsep}{0em}
		\item Download the source archive to the Link
		\item Request a remote compilation on the Link.
		\item The Link extracts the archive to a temporary directory
		\item The Link invokes pcompiler on the directory and blocks until
		the compilation is completed.
		\item KISS IDE then receives these results from the Link.
		\item Results are formatted, highlighted, and then presented to the user.
	\end{enumerate}
	
	\why{While compile time is negatively influenced by this methodology, there are several notable advantages:
	\begin{itemize}
		\setlength{\itemsep}{0em}
		\item A firmware exposing new APIs on the Link doesn't require a new version of KISS IDE. This was a large problem
		with the CBCv2 robotics controller.
		\item KISS IDE can work without a local compiler.
		\item The archive format is language neutral, which means languages can be added or removed without
		changing KISS IDE (although syntax highlighting might not work).
	\end{itemize}}
	
	\getit{\url{http://github.com/kipr/kiss}}
	\section{Building Firmwares and Packages}
	There is unfortunately no easy way to build firmwares and packages for the Link at this time. This section will, however, describe how
	to do it the hard way (and the way KIPR does it internally). The Link is based off of a linux distribution system called \emph{OpenEmbedded}
	and its associated build system called \texttt{bitbake}. While these tools have improved dramatically in recent months and years, the Link's
	build system has not been updated yet. As such, there will be little to no relevant documentation available to assist in the creation
	of packages or firmwares, and most software available in OpenEmbedded's local repository will be outdated. Fortunately, KIPR has shed most of
	the tears of frustration on the reader's behalf. What follows is a guide on setting up a build server and modifying pieces of KIPR's software.
	
	\subsection{Prerequisites}
	\note{These instructions require a good amount of experience with linux and distributed version control systems. If you haven't had
	any experience with those, it is recommended you gain knowledge of them \emph{before} rather than \emph{during} this process.}
	
	The following software and hardware is necessary for setting up a Link-compatible OpenEmbedded build server (herein referred to as
	\emph{the host}):
	
	\begin{enumerate}
		\setlength{\itemsep}{0em}
		\item A computer or virtual machine running a \emph{clean install} of the latest Ubuntu Linux. This computer or virtual machine must have:
		\begin{enumerate}
			\setlength{\itemsep}{0em}
			\item At least 150 GB of free space (though 300 GB or more is preferred).
			\item At least 4 GB of RAM. KIPR uses 16 GB of RAM.
			\item At least a quad-core CPU. KIPR uses a six-core AMD CPU, but a newer Intel quad-core
			chip would probably be more performant.
			\item A broadband internet connection.
		\end{enumerate}
		\item Knowledge of the \texttt{git} version control system.
		\item Fluent in basic and moderately complex linux terminal commands.
	\end{enumerate}
	
	Once Ubuntu Linux has been installed on the host:
	
	\texttt{
\$ sudo apt-get install git subversion binutils
	}
	
	\textbf{Insert Clemens' instructions once they are ready}
	
	\subsection{Core Concepts in OpenEmbedded and \texttt{bitbake}}
	OpenEmbedded organizes build \emph{recipes} into ``meta'' \emph{layers}. 
	
	\subsection{Modifying an Existing Package}
	
	First, change the working directory to \texttt{oe} using \texttt{\$ cd oe}. Once inside the \texttt{oe} directory, source in the build
	environment using the command \texttt{\$ . $\sim$/.oe/environment-oecore}. This step must be done every time a new terminal is used.
	
	To tell bitbake to pull new code from a given package's repository, the PR version of the package must be incremented. For example,
	to increment the PR version of botui using vim, one could do the following:
	
	\noindent \texttt{\$ vim sources/meta-kipr/recipes-kovan/botui/botui\_git.bb} \\
	\noindent Then enter the characters: \texttt{?}, \texttt{P}, \texttt{R}, \texttt{[Enter]}, \texttt{Ctrl-A}, \texttt{:}, \texttt{w}, \texttt{q}, \texttt{[Enter]}.
	
	\novice{Vim is an advanced text editor that can be used to modify files in a terminal. The first part of this command is \texttt{?PR[Enter]}.
	This tells vim to find the first occurrence of the string ``PR'' in the text file and move the cursor to it. \texttt{Ctrl-A} tells vim to
	find the next integer (in this case the PR number) and increment it by 1. Finally, \texttt{:wq[Enter]} tells vim to write its changes to disk
	and then quit.}
	
	Now, rebuild the package using \texttt{\$ bitbake botui}
	
	
	\subsection{A Custom Package}
	
	\subsection{Debugging Bitbake Errors}
	Bitbake will sometimes simply fail with a message containing the word ``ERROR''. Since there are thousands of things ``ERROR'' could
	mean, this will often leave the developer scratching their heads in frustration. I have discovered, however, that most of these ``ERROR''
	messages with zero information are often the result of a syntax error in a \texttt{.bb} package file. As such, it is strongly recommended
	that developers incrementally test changes to their \texttt{.bb} files. If that doesn't help, the developer will need to go through their
	recently modified \texttt{.bb} files and temporally remove them from the repository to check if that file is indeed the problem. Once the
	problematic file has been identified, comment out portions of the file to determine wherein the error lies.
	
	\subsection{Installing Packages and Building Firmwares}
	OpenEmbedded will automatically generate package files that are installable on the Link using a flash drive and terminal.
	
	\subsection{Developing Without a Build Server}
	Several of the aforementioned software packages will compile and run on a regular computer.
	
	\section{Essential Tools for Working with the Link}
	
	\subsection{Secure Shell (SSH)}
	\note{As of firmware 1.9.8, \texttt{botui} sets the root's password to the password configured in
	Home$\rightarrow$Settings$\rightarrow$Communication.}
	
	\texttt{ssh} allows root-level access to the Link over a Wi-Fi connection and is often the most convenient method of
	modifying the device. First connect the Link to a Wi-Fi network.
	
	\subsection{\texttt{opkg} Package Manager}
	The Link's default package manager is called \texttt{opkg}. \texttt{opkg} can be used to install packages built using
	a Link build server.
	
	\section{Modification Checklists}
	When creating a component for the Link, there are often several steps that are often obscure and non-obvious. These checklists
	hope create a discrete list of steps for common Link modifications a developer might want to perform.
	
	\subsection{Adding Support For a New Language}
	\begin{itemize}
		\setlength{\itemsep}{0em}
		\renewcommand{\labelitemi}{$\square$} 
		\item Write the required \texttt{bitbake} \emph{recipes} to create binary packages for the language's compiler and libraries.
		\item Write the necessary \texttt{pcompiler} \emph{transforms} to compile the language into a binary.
		\item Add the language's file extension to \texttt{botui}'s \texttt{FileActionCompile} class to enable compiling from
		the Link's embedded file manager.
		\item Write a \texttt{Lexer} (syntax highlighter) plugin for KISS IDE.
		\item Create a \emph{Template Pack} for KISS IDE (File$\rightarrow$New$\rightarrow$New Template Pack...).
		\item Add the required language packages to \texttt{meta-kipr/recipes-kovan/}\\\texttt{images/kovan-kipr-image.bb} to make the language
		a part of the official firmware.
	\end{itemize}
	
	\subsection{}
	\begin{itemize}
		\setlength{\itemsep}{0em}
		\renewcommand{\labelitemi}{$\square$} 
		\item Write the required \texttt{bitbake} \emph{recipes} to create binary packages for the language's compiler and libraries.
		\item Write the necessary \texttt{pcompiler} \emph{transforms} to compile the language into a binary.
		\item Write a \texttt{Lexer} (syntax highlighter) plugin for KISS IDE.
		\item Create a \emph{Template Pack} for KISS IDE (File$\rightarrow$New$\rightarrow$New Template Pack...).
	\end{itemize}
	
	
	
	\section{Conclusion}
	While this document covers several pieces of the Link's firmware
	
	
	\subsection{Acknowledgements}
	\begin{itemize}
		\setlength{\itemsep}{0em}
		\item Thank you to Nafis Zaman for his contributions to \texttt{botui} and KISS IDE.
		\item Thank you to Clemens Koza for his efforts in creating reliable instructions for setting up a build server from scratch.
		\item Thank you to the users that have coped with the early instabilities of the new system.
	\end{itemize}

	
	\section{Appendix}
	
	\subsection{Manually Un-bricking a Link}
	\noindent \textbf{These steps will void your warranty.}
	
	\noindent If the Link fails to boot to standard mode or recovery mode, these steps can be used to manually flash the internal micro-SD card. The following steps require a Torx screwdriver, a micro-SD card reader (or SD card reader and adapter), and an Operating System supporting \texttt{dd} (*nix or Mac OS X).
	
	\begin{enumerate}
		\setlength{\itemsep}{0em}
		\item Remove the warranty sticker from the Link's case.
		\item Remove the four bottom screws using a Torx screwdriver.
		\item Locate the micro-SD card hinge (by the power switch).
		\item Push the hinge towards the inside of the case and lift up.
		\item Place the recovered micro-SD card in card reader.
		\item The OS will probably auto-mount the contents of the drive and
		possibly display some error messages. Ignore these error messages and
		unmount the SD card if auto-mounted.
		\item Determine the appropriate device file to write to in \texttt{/dev}.
		This can easily be determined by removing the card,
		running \texttt{\$ ls /dev > 1}, inserting the card again, 
		and running \texttt{\$ ls /dev > 2 \&\& diff 1 2}. The correct
		device file will not have any numerical suffix.
		\item Download the KIPR Link firmware file to install.
		\item Extract the KIPR Link firmware file using \texttt{gunzip}.
	\end{enumerate}
	
	\whoathere{Triple check that the determined device file is indeed the micro-SD card.
	Choosing the wrong device file could overwrite your hard drive disk!}
	
	\noindent Finally, \texttt{\$ sudo dd if=firmware\_file.img of=/dev/determined\_device\_file bs=1M}
		(the M might be lowercase on your platform).
	
	\vspace{\fill}
	\noindent {\small This document was compiled on \today.}
	
\end{document}